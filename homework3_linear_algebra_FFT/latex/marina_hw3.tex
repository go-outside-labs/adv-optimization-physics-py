% MIA STEIN, SPRING/2013
% http://mysbfiles.stonybrook.edu/~mvonsteinkir/

%%%%%%%%%%%%%%%%%%%%%%%%%%%%%%%%%%%%%%%%%%%%%%%%%%%%%%%%%%%%%%%%%%%%%%%%%%%%%%%%%%%%%%%%%%%%%%
%%%%%%%%%%%%%%%%%%%%%%%%%%%%%%%%%%%%%		Header		%%%%%%%%%%%%%%%%%%%%%%%%%%%%%%%%%%%%%%
%%%%%%%%%%%%%%%%%%%%%%%%%%%%%%%%%%%%%%%%%%%%%%%%%%%%%%%%%%%%%%%%%%%%%%%%%%%%%%%%%%%%%%%%%%%%%%

\documentclass[10pt]{article}

\usepackage{epsfig}
\usepackage{color}
\usepackage[usenames,dvipsnames,svgnames,table]{xcolor}
\usepackage{indentfirst}
\usepackage{fancyhdr, lastpage}

\setlength{\oddsidemargin}{0in}
\setlength{\textwidth}{6.5in}

\pagestyle{myheadings}
\pagestyle{fancy}

\newcounter{question}[section]
\newcommand{\question}[2] {\vspace{.25in} \fbox{#1} #2 \vspace{.10in}}
\renewcommand{\part}[1] {\vspace{.10in} {\bf (#1)}}
\newcommand{\ie}{{\it i.e., }}
\newcommand{\eg}{{\it e.g., }}

\markright{PHYS 688: Homework \#3 Numerical Methods for (Astro)Physics}
\author{\color{purple}{\bf MIA STEIN}}
\title{\color{red}{\bf PHYS 688: Numerical Methods for AstroPhysics \\Homework \#3: Linear Algebra and FFTs} }

\fancyhf{}
\lhead{MIA STEIN}
\chead{Linear Algebra and FFTs}
\rhead{Homework \#3}
\cfoot{Page \thepage{} of \protect\pageref{LastPage}}


%%%%%%%%%%%%%%%%%%%%%%%%%%%%%%%%%%%%%%%%%%%%%%%%%%%%%%%%%%%%%%%%%%%%%%%%%%%%%%%%%%%%%%%%%%%%%%
%%%%%%%%%%%%%%%%%%%%%%%%%%%%%%%%%%%%%%%%%%%%%%%%%%%%%%%%%%%%%%%%%%%%%%%%%%%%%%%%%%%%%%%%%%%%%%

\begin{document}
\maketitle



{\color{MidnightBlue}
\question{Q.1}
{}{\color{MidnightBlue}
\question{Q.2}
{} \texttt{{\bf See outputs from source codes. }}}}


{\color{MidnightBlue}
\question{Q.3}{\texttt{{\bf(Time Series)} {\bf The equation of motion for a damped driven pendulum is:
$$ml \frac{d\theta^2}{dt^2} = F_g + F_d + F_r,$$
where $\theta$ is the angle of the pendulum from vertical, $F_r = f_0 \cos(\omega_d)$ is the driving force with amplitude $f_0$ and driving frequency $\omega_d$, $F_d = \kappa \nu$ is the resistive force and $F_g = mg \sin (\theta)$ is the gravitational force on the pendulum. We can write this as a system of the form:
$$\frac{\theta}{dt} = \omega,$$
$$\frac{\omega}{dt}˙ = -q \omega - \sin(\theta) + b \cos (\omega_d),$$
where $q$ is a {\it scaled damping coefficient} and $b$ is a {\it scaled forcing amplitude}. We also took $l = g$ to simplify things.}}}} 




\begin{enumerate}

\item[1.] We integrate this system using {\it 4th-order Runge-Kutta} and a {\it uniform timestep} (equally spaced points to do Fourier analysis). We chose the parameters $q = 0.5$, $b = 0.9$, and $\omega_d=2/3$. We integrate for tmax = 100 periods and we pick a timestep dt = 0.05 that gives a converged solution. We plot $\omega - \theta$ (left top plots in the below figures) and we can see the period motion for $0.5 \leq b<10.0$.

\item[2.] We perform the {\it discrete Fourier transform} (DFT) of $\theta(t)$ and plot the power spectrum for values $0.5 \leq b\leq 13.5$ (figures above), where for $b\geq 10.0$ the behaviour is chaotic.

\end{enumerate}



\begin{figure} [ht]
\begin{center}
\includegraphics[scale=0.6]{plots/b-05_phase_space.png} \\
\includegraphics[scale=0.6]{plots/b-05_dft.png} 
\caption{{\bf Periodic Regime, b = 0.5}: (left) {\it Phase space} and {\it power spectrum} for the damped driven  pendulum. (right) The {\it discrete Fourier transform} (DFT) and its inverse, returning to the original function).}
\end{center}
\end{figure}

\begin{figure} [ht]
\begin{center}
\includegraphics[scale=0.6]{plots/b-055_phase_space.png} \\
\includegraphics[scale=0.6]{plots/b-055_dft.png} 
\caption{{\bf Periodic Regime, b = 0.55}: (left) {\it Phase space} and {\it power spectrum} for the damped driven  pendulum. (right) The {\it discrete Fourier transform} (DFT) and its inverse, returning to the original function).}
\end{center}
\end{figure}

\begin{figure} [ht]
\begin{center}
\includegraphics[scale=0.6]{plots/b-06_phase_space.png} \\
\includegraphics[scale=0.6]{plots/b-06_dft.png} 
\caption{{\bf Periodic Regime, b = 0.6}: (left) {\it Phase space} and {\it power spectrum} for the damped driven  pendulum. (right) The {\it discrete Fourier transform} (DFT) and its inverse, returning to the original function).}
\end{center}
\end{figure}

\begin{figure} [ht]
\begin{center}
\includegraphics[scale=0.6]{plots/b-065_phase_space.png} \\
\includegraphics[scale=0.6]{plots/b-065_dft.png} 
\caption{{\bf Periodic Regime, b = 0.65}: (left) {\it Phase space} and {\it power spectrum} for the damped driven  pendulum. (right) The {\it discrete Fourier transform} (DFT) and its inverse, returning to the original function).}
\end{center}
\end{figure}

\begin{figure} [ht]
\begin{center}
\includegraphics[scale=0.6]{plots/b-07_phase_space.png} \\
\includegraphics[scale=0.6]{plots/b-07_dft.png} 
\caption{{\bf Periodic Regime, b = 0.07}: (left) {\it Phase space} and {\it power spectrum} for the damped driven  pendulum. (right) The {\it discrete Fourier transform} (DFT) and its inverse, returning to the original function).}
\end{center}
\end{figure}

\begin{figure} [ht]
\begin{center}
\includegraphics[scale=0.6]{plots/b-075_phase_space.png} \\
\includegraphics[scale=0.6]{plots/b-055_dft.png} 
\caption{{\bf Periodic Regime, b = 0.75}: (left) {\it Phase space} and {\it power spectrum} for the damped driven  pendulum. (right) The {\it discrete Fourier transform} (DFT) and its inverse, returning to the original function).}
\end{center}
\end{figure}

\begin{figure} [ht]
\begin{center}
\includegraphics[scale=0.6]{plots/b-08_phase_space.png} \\
\includegraphics[scale=0.6]{plots/b-08_dft.png} 
\caption{{\bf Periodic Regime, b = 0.8}: (left) {\it Phase space} and {\it power spectrum} for the damped driven  pendulum. (right) The {\it discrete Fourier transform} (DFT) and its inverse, returning to the original function).}
\end{center}
\end{figure}

\begin{figure} [ht]
\begin{center}
\includegraphics[scale=0.6]{plots/b-085_phase_space.png} \\
\includegraphics[scale=0.6]{plots/b-085_dft.png} 
\caption{{\bf Periodic Regime, b = 0.85}: (left) {\it Phase space} and {\it power spectrum} for the damped driven  pendulum. (right) The {\it discrete Fourier transform} (DFT) and its inverse, returning to the original function).}
\end{center}
\end{figure}


\begin{figure} [ht]
\begin{center}
\includegraphics[scale=0.6]{plots/b-09_phase_space.png} \\
\includegraphics[scale=0.6]{plots/b-09_dft.png} 
\caption{{\bf Periodic Regime, b = 0.9}: (left) {\it Phase space} and {\it power spectrum} for the damped driven  pendulum. (right) The {\it discrete Fourier transform} (DFT) and its inverse, returning to the original function).}
\end{center}
\end{figure}

\begin{figure} [ht]
\begin{center}
\includegraphics[scale=0.6]{plots/b-095_phase_space.png} \\
\includegraphics[scale=0.6]{plots/b-095_dft.png} 
\caption{{\bf Periodic Regime, b = 0.95}: (left) {\it Phase space} and {\it power spectrum} for the damped driven  pendulum. (right) The {\it discrete Fourier transform} (DFT) and its inverse, returning to the original function).}
\end{center}
\end{figure}


\quad


\begin{figure} [ht]
\begin{center}
\includegraphics[scale=0.6]{plots/b-10_phase_space.png} \\
\includegraphics[scale=0.6]{plots/b-10_dft.png} 
\caption{{\bf Chaotic Regime, b = 1.0}: (left) {\it Phase space} and {\it power spectrum} for the damped driven  pendulum. (right) The {\it discrete Fourier transform} (DFT) and its inverse, returning to the original function).}
\end{center}
\end{figure}

\begin{figure} [ht]
\begin{center}
\includegraphics[scale=0.6]{plots/b-105_phase_space.png} \\
\includegraphics[scale=0.6]{plots/b-105_dft.png} 
\caption{{\bf Chaotic Regime, b = 10.5}: (left) {\it Phase space} and {\it power spectrum} for the damped driven  pendulum. (right) The {\it discrete Fourier transform} (DFT) and its inverse, returning to the original function).}
\end{center}
\end{figure}

\begin{figure} [ht]
\begin{center}
\includegraphics[scale=0.6]{plots/b-11_phase_space.png} \\
\includegraphics[scale=0.6]{plots/b-11_dft.png} 
\caption{{\bf Chaotic Regime, b = 11.1}: (left) {\it Phase space} and {\it power spectrum} for the damped driven  pendulum. (right) The {\it discrete Fourier transform} (DFT) and its inverse, returning to the original function).}
\end{center}
\end{figure}

\begin{figure} [ht]
\begin{center}
\includegraphics[scale=0.6]{plots/b-115_phase_space.png} \\
\includegraphics[scale=0.6]{plots/b-115_dft.png} 
\caption{{\bf Chaotic Regime, b = 11.5}: (left) {\it Phase space} and {\it power spectrum} for the damped driven  pendulum. (right) The {\it discrete Fourier transform} (DFT) and its inverse, returning to the original function).}
\end{center}
\end{figure}

\begin{figure} [ht]
\begin{center}
\includegraphics[scale=0.6]{plots/b-12_phase_space.png} \\
\includegraphics[scale=0.6]{plots/b-12_dft.png} 
\caption{{\bf Chaotic Regime, b = 12}: (left) {\it Phase space} and {\it power spectrum} for the damped driven  pendulum. (right) The {\it discrete Fourier transform} (DFT) and its inverse, returning to the original function).}
\end{center}
\end{figure}

\begin{figure} [ht]
\begin{center}
\includegraphics[scale=0.6]{plots/b-125_phase_space.png} \\
\includegraphics[scale=0.6]{plots/b-125_dft.png} 
\caption{{\bf Chaotic Regime, b = 12.5}: (left) {\it Phase space} and {\it power spectrum} for the damped driven  pendulum. (right) The {\it discrete Fourier transform} (DFT) and its inverse, returning to the original function).}
\end{center}
\end{figure}


\begin{figure} [ht]
\begin{center}
\includegraphics[scale=0.6]{plots/b-13_phase_space.png} \\
\includegraphics[scale=0.6]{plots/b-13_dft.png} 
\caption{{\bf Chaotic Regime, b = 13}: (left) {\it Phase space} and {\it power spectrum} for the damped driven  pendulum. (right) The {\it discrete Fourier transform} (DFT) and its inverse, returning to the original function).}
\end{center}
\end{figure}

\begin{figure} [ht]
\begin{center}
\includegraphics[scale=0.6]{plots/b-135_phase_space.png} \\
\includegraphics[scale=0.6]{plots/b-135_dft.png} 
\caption{{\bf Chaotic Regime, b = 13.5}: (left) {\it Phase space} and {\it power spectrum} for the damped driven  pendulum. (right) The {\it discrete Fourier transform} (DFT) and its inverse, returning to the original function).}
\end{center}
\end{figure}

\quad



%%%%%%%%%

\end{document}

